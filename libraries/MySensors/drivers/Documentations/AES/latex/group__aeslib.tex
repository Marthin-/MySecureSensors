\hypertarget{group__aeslib}{}\section{A\+ES library for Arduino and Raspberry pi}
\label{group__aeslib}\index{A\+E\+S library for Arduino and Raspberry pi@{A\+E\+S library for Arduino and Raspberry pi}}
\hypertarget{group__aeslib_AesGoals}{}\subsection{design Goals}\label{group__aeslib_AesGoals}
This library is A\+E\+Signed to be... \begin{DoxyItemize}
\item Fast and efficient. \item Able to effectively encrypt and decrypt any size of string. \item Able to encrypt and decrypt using \hyperlink{classAES}{A\+ES} \item Able to encrypt and decrypt using A\+E\+S-\/\+C\+BC \item Easy for the user to use in his programs.\end{DoxyItemize}
\hypertarget{group__aeslib_Acknowledgements}{}\subsection{Acknowledgements}\label{group__aeslib_Acknowledgements}
This is an \hyperlink{classAES}{A\+ES} library for the Arduino, based on tzikis\textquotesingle{}s \hyperlink{classAES}{A\+ES} library, which you can find \href{https://github.com/tzikis/arduino}{\tt here\+:}.~\newline
 Tzikis library was based on scottmac`s library, which you can find \href{https://github.com/scottmac/arduino}{\tt here\+:}~\newline
\hypertarget{group__aeslib_Installation}{}\subsection{Installation}\label{group__aeslib_Installation}
\paragraph*{Arduino}

Create a folder named {\itshape \hyperlink{classAES}{A\+ES}} in the {\itshape libraries} folder inside your Arduino sketch folder. If the libraries folder doesn\textquotesingle{}t exist, create it. Then copy everything inside. (re)launch the Arduino I\+DE.~\newline
 You\textquotesingle{}re done. Time for a mojito

\paragraph*{Raspberry pi}

{\bfseries install}~\newline
~\newline


sudo make install~\newline
 cd examples\+\_\+\+Rpi~\newline
 make~\newline
~\newline


{\bfseries What to do after changes to the library}~\newline
~\newline
 sudo make clean~\newline
 sudo make install~\newline
 cd examples\+\_\+\+Rpi~\newline
 make clean~\newline
 make~\newline
~\newline
 {\bfseries What to do after changes to a sketch}~\newline
~\newline
 cd examples\+\_\+\+Rpi~\newline
 make $<$sketch$>$~\newline
~\newline
 or ~\newline
 make clean~\newline
 make~\newline
~\newline
~\newline
 {\bfseries How to start a sketch}~\newline
~\newline
 cd examples\+\_\+\+Rpi~\newline
 sudo ./$<$sketch$>$~\newline
~\newline
\hypertarget{group__aeslib_AesNews}{}\subsection{News}\label{group__aeslib_AesNews}
If issues are discovered with the documentation, please report them \href{https://github.com/spaniakos/spaniakos.github.io/issues}{\tt here} \hypertarget{group__aeslib_AesUseful}{}\subsection{Useful References}\label{group__aeslib_AesUseful}
Please refer to\+:

\begin{DoxyItemize}
\item \href{http://spaniakos.github.io/AES/classAES.html}{\tt {\bfseries A\+ES} Class Documentation} \item \href{https://github.com/spaniakos/AES/archive/master.zip}{\tt {\bfseries Download}} \item \href{https://github.com/spaniakos/AES/}{\tt {\bfseries Source Code}} \item \href{http://spaniakos.github.io/}{\tt All spaniakos Documentation Main Page}\end{DoxyItemize}
\hypertarget{group__aeslib_AesBoard_Support}{}\subsection{Board Support}\label{group__aeslib_AesBoard_Support}
Most standard Arduino based boards are supported\+:
\begin{DoxyItemize}
\item Arduino
\item Intel Galileo support
\item Raspberry Pi Support
\item The library has not been tested to other boards, but it should suppport A\+T\+Mega 328 based boards,Mega Boards,Arduino Due,A\+T\+Tiny board 
\end{DoxyItemize}