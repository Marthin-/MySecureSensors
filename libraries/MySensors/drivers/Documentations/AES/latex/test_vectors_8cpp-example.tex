\hypertarget{test_vectors_8cpp-example}{}\section{test\+\_\+vectors.\+cpp}
{\bfseries For Rasberry pi}~\newline
 {\bfseries Updated\+: spaniakos 2015 }~\newline


This is an example of monte carlo test vectors, in order to justify the effectiveness of the algorithm.~\newline
 plus is a classical approach to the \hyperlink{classAES}{A\+ES} encryption library with out the easy to use add-\/on modifications.


\begin{DoxyCodeInclude}
\textcolor{preprocessor}{#include <AES.h>}
\textcolor{preprocessor}{#include "printf.h"}

\hyperlink{classAES}{AES} aes ;

byte key [2*N\_BLOCK] ;
byte plain [N\_BLOCK] ;
byte iv [N\_BLOCK] ;
byte cipher [N\_BLOCK] ;
byte check [N\_BLOCK] ;

\textcolor{keywordtype}{void} print\_value (\textcolor{keyword}{const} \textcolor{keywordtype}{char} * str, byte * a, \textcolor{keywordtype}{int} bits);
\textcolor{keywordtype}{void} prekey\_test\_var\_plaintext (\textcolor{keywordtype}{int} bits);
\textcolor{keywordtype}{void} prekey\_test\_var\_key (\textcolor{keywordtype}{int} bits);
\textcolor{keywordtype}{void} set\_bits (\textcolor{keywordtype}{int} bits, byte * a, \textcolor{keywordtype}{int} count);
\textcolor{keywordtype}{void} check\_same (byte * a, byte * b, \textcolor{keywordtype}{int} bits);
\textcolor{keywordtype}{void} monte\_carlo (\textcolor{keywordtype}{int} bits);

\textcolor{keywordtype}{int} main(\textcolor{keywordtype}{int} argc, \textcolor{keywordtype}{char}** argv)
\{
  printf (\textcolor{stringliteral}{"AES library test vectors"}) ;

  monte\_carlo (128) ;

  \textcolor{keywordflow}{for} (\textcolor{keywordtype}{int} keysize = 128 ; keysize <= 256 ; keysize += 64)
    \{
      prekey\_test\_var\_plaintext (keysize) ;
      prekey\_test\_var\_key (keysize) ;
    \}
\}


\textcolor{keywordtype}{void} prekey\_test\_var\_plaintext (\textcolor{keywordtype}{int} bits)
\{
  printf (\textcolor{stringliteral}{"\(\backslash\)nECB Varying Plaintext %i bits\(\backslash\)n"},bits) ;
 
  byte succ ;
  set\_bits (bits, key, 0) ;  \textcolor{comment}{// all zero key}
  succ = aes.\hyperlink{classAES_afe2900d46f475f6f3ea8d164e1581ed9}{set\_key} (key, bits) ;
  \textcolor{keywordflow}{if} (succ != SUCCESS)
    printf(\textcolor{stringliteral}{"Failure set\_key\(\backslash\)n"}) ;


  \textcolor{keywordflow}{for} (\textcolor{keywordtype}{int} bitcount = 1 ; bitcount <= 128 ; bitcount++)
    \{
      printf (\textcolor{stringliteral}{"COUNT = %i \(\backslash\)n"},bitcount-1);
      print\_value (\textcolor{stringliteral}{"KEY = "}, key, bits) ;
      set\_bits (128, plain, bitcount) ;
      
      print\_value (\textcolor{stringliteral}{"PLAINTEXT = "}, plain, 128) ;
      
      succ = aes.\hyperlink{classAES_a72a674e99a92e296d1bf03444fe6ea15}{encrypt} (plain, cipher) ;
      \textcolor{keywordflow}{if} (succ != SUCCESS)
        printf (\textcolor{stringliteral}{"Failure encrypt\(\backslash\)n"}) ;

      print\_value (\textcolor{stringliteral}{"CIPHERTEXT = "}, cipher, 128) ;
      
      succ = aes.\hyperlink{classAES_abc514d1129789a60d60127f151450e9c}{decrypt} (cipher, check) ;
      \textcolor{keywordflow}{if} (succ != SUCCESS)
        printf (\textcolor{stringliteral}{"Failure decrypt\(\backslash\)n"}) ;

      \textcolor{comment}{//print\_value ("CHECK = ", check, 128) ;}
      check\_same (plain, check, 128) ;
      printf (\textcolor{stringliteral}{"\(\backslash\)n"}) ;
    \}
\}


\textcolor{keywordtype}{void} prekey\_test\_var\_key (\textcolor{keywordtype}{int} bits)
\{
  printf (\textcolor{stringliteral}{"\(\backslash\)nECB Varying key %i bits\(\backslash\)n"},bits);
  
  byte succ ;
  set\_bits (128, plain, 0) ;

  \textcolor{keywordflow}{for} (\textcolor{keywordtype}{int} bitcount = 1 ; bitcount <= bits ; bitcount++)
    \{
      set\_bits (bits, key, bitcount) ;  \textcolor{comment}{// all zero key}
      succ = aes.\hyperlink{classAES_afe2900d46f475f6f3ea8d164e1581ed9}{set\_key} (key, bits) ;
      \textcolor{keywordflow}{if} (succ != SUCCESS)
        printf (\textcolor{stringliteral}{"Failure set\_key\(\backslash\)n"}) ;
      printf (\textcolor{stringliteral}{"COUNT = %i\(\backslash\)n"},bitcount-1);
      print\_value (\textcolor{stringliteral}{"KEY = "}, key, bits) ;

      print\_value (\textcolor{stringliteral}{"PLAINTEXT = "}, plain, 128) ;

      succ = aes.\hyperlink{classAES_a72a674e99a92e296d1bf03444fe6ea15}{encrypt} (plain, cipher) ;
      \textcolor{keywordflow}{if} (succ != SUCCESS)
        printf (\textcolor{stringliteral}{"Failure encrypt\(\backslash\)n"}) ;

      print\_value (\textcolor{stringliteral}{"CIPHERTEXT = "}, cipher, 128) ;

      succ = aes.\hyperlink{classAES_abc514d1129789a60d60127f151450e9c}{decrypt} (cipher, check) ;
      \textcolor{keywordflow}{if} (succ != SUCCESS)
        printf (\textcolor{stringliteral}{"Failure decrypt\(\backslash\)n"}) ;

      check\_same (plain, check, 128) ;
      printf(\textcolor{stringliteral}{"\(\backslash\)n"});
    \}
\}

\textcolor{keywordtype}{void} set\_bits (\textcolor{keywordtype}{int} bits, byte * a, \textcolor{keywordtype}{int} count)
\{
  bits >>= 3 ;
  byte bcount = count >> 3 ;
  \textcolor{keywordflow}{for} (byte i = 0 ; i < bcount ; i++)
    a [i] = 0xFF ;
  \textcolor{keywordflow}{if} ((count & 7) != 0)
    a [bcount++] = 0xFF & (0xFF00 >> (count & 7)) ;
  \textcolor{keywordflow}{for} (byte i = bcount ; i < bits ; i++)
    a [i] = 0x00 ;
\}

\textcolor{keywordtype}{void} check\_same (byte * a, byte * b, \textcolor{keywordtype}{int} bits)
\{
  bits >>= 3 ;
  \textcolor{keywordflow}{for} (byte i = 0 ; i < bits ; i++)
    \textcolor{keywordflow}{if} (a[i] != b[i])
      \{
        printf (\textcolor{stringliteral}{"Failure plain != check\(\backslash\)n"}) ;
        return ;
      \}
\}

\textcolor{keywordtype}{char} hex[] = \textcolor{stringliteral}{"0123456789abcdef"} ;


\textcolor{keywordtype}{void} print\_value (\textcolor{keyword}{const} \textcolor{keywordtype}{char} * str, byte * a, \textcolor{keywordtype}{int} bits)
\{
  printf (\textcolor{stringliteral}{"%s"},str) ;
  bits >>= 3 ;
  \textcolor{keywordflow}{for} (\textcolor{keywordtype}{int} i = 0 ; i < bits ; i++)
    \{
      byte b = a[i] ;
      printf(\textcolor{stringliteral}{"%c%c"},hex[b >> 4],hex [b & 15]);
      \textcolor{comment}{//Serial.print (hex [b >> 4]) ;}
      \textcolor{comment}{//Serial.print (hex [b & 15]) ;}
    \}
  printf(\textcolor{stringliteral}{"\(\backslash\)n"}) ;
\}

byte monteplain [] = 
  \{ 0xb9, 0x14, 0x5a, 0x76, 0x8b, 0x7d, 0xc4, 0x89, 
    0xa0, 0x96, 0xb5, 0x46, 0xf4, 0x3b, 0x23, 0x1f \} ;
byte montekey []   = 
  \{ 0x13, 0x9a, 0x35, 0x42, 0x2f, 0x1d, 0x61, 0xde, 
    0x3c, 0x91, 0x78, 0x7f, 0xe0, 0x50, 0x7a, 0xfd \} ;

\textcolor{keywordtype}{void} monte\_carlo (\textcolor{keywordtype}{int} bits)
\{
  printf (\textcolor{stringliteral}{"\(\backslash\)nMonte Carlo %i bits\(\backslash\)n"},bits);
  byte succ;
  \textcolor{keywordflow}{for} (\textcolor{keywordtype}{int} i = 0 ; i < 16 ; i++)
  \{
    plain [i] = monteplain [i] ;
    key [i] = montekey [i] ;
  \}
  \textcolor{keywordflow}{for} (\textcolor{keywordtype}{int} i = 0 ; i < 100 ; i++)
    \{
     printf (\textcolor{stringliteral}{"COUNT = %i\(\backslash\)n"},i);
      print\_value (\textcolor{stringliteral}{"KEY = "}, key, bits) ;
      print\_value (\textcolor{stringliteral}{"PLAINTEXT = "}, plain, 128) ;
      succ = aes.\hyperlink{classAES_afe2900d46f475f6f3ea8d164e1581ed9}{set\_key} (key, bits) ;
      \textcolor{keywordflow}{for} (\textcolor{keywordtype}{int} j = 0 ; j < 1000 ; j++)
        \{
          succ = aes.\hyperlink{classAES_a72a674e99a92e296d1bf03444fe6ea15}{encrypt} (plain, cipher) ;
          aes.\hyperlink{classAES_ae7b28053b759aea46d13f2c8ebc3b64a}{copy\_n\_bytes} (plain, cipher, 16) ;
        \}
      print\_value (\textcolor{stringliteral}{"CIPHERTEXT = "}, cipher, 128) ;
      printf(\textcolor{stringliteral}{"\(\backslash\)n"});
      \textcolor{keywordflow}{if} (bits == 128)
        \{
          \textcolor{keywordflow}{for} (byte k = 0 ; k < 16 ; k++)
            key [k] ^= cipher [k] ;
        \}
      \textcolor{keywordflow}{else} \textcolor{keywordflow}{if} (bits == 192)
        \{
        \}
      \textcolor{keywordflow}{else}
        \{
        \}
    \}
\}
\end{DoxyCodeInclude}
 